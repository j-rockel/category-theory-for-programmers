\documentclass[12pt]{article}
\setlength\parindent{0pt}

\title{Example Categories\\
\large from ``Category Theory for Programmers" by  Bartosz Milewski}

\author{Jessika Rockel}
\date{\today}

\begin{document}
\maketitle

\section{Chapter 2: Types and Functions}
\subsection{Set}
The category of sets, where objects are sets and morphisms are functions. 
Here we can derive a lot of information about the category from the information we have about sets and the possible functions between them. 
This is in contrast to most other categories, where we are not interested in looking ``inside'' objects or morphisms.

\subsection{Hask}
The category of Haskell types. \textbf{Set} with the addition of \textit{bottom} ($\bot$) to represent nontermination. \\(There's more to this but it is not included in the chapter).

\section{Chapter 3: Categories Great and Small}
\subsection{No Objects}
A category that has zero objects and therefore also zero morphisms. 

\subsection{Free Category from Directed Graph}
A \textit{free category} can be generated from any directed graph by adding identity and composition arrows everywhere possible (which corresponds to performing the reflexive and transitive closure for the graph's edge relation).

\subsection{Orders}
The category where a morphism is a relation between objects works for any relation that is at least a preorder: A preorder is reflexive (includes identity) and transitive (includes composition) and thus fulfills exactly the requirements of a category. One such example is the less than or equal relation $a \leq b$.

\subsection{Monoid}
A categorical monoid, which is a one-element category. 
\end{document}